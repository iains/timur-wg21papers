\input{wg21common}

\begin{document}
\title{Contracts on lambdas}
\author{ Timur Doumler \small(\href{mailto:papers@timur.audio}{papers@timur.audio})}
\date{}
\maketitle

\begin{tabular}{ll}
Document \#: & P2890R1 \\
Date: &2023-11-06 \\
Project: & Programming Language C++ \\
Audience: & SG21
\end{tabular}

\begin{abstract}
This paper proposes to allow precondition and postcondition checks on lambda expressions. We propose a rule for name lookup and discuss two alternatives for specifying the semantics of lambda captures in such checks, one that prioritises simplicity and consistency with the rest of the language and another that prioritises the zero overhead principle from \cite{P2932R1}.
\end{abstract}

\section{Introduction}
\label{sec:intro}

In the Contracts MVP (see \cite{P2900R1}), preconditions and postconditions are so far only allowed on ordinary functions. However, there is no a priori reason why we should not allow them to appear also on lambda expressions. For example, the following code should be well-formed:
\begin{codeblock}
constexpr bool add_overflows(int a, int b) {
  return (b > 0 && a > INT_MAX - b) || (b < 0 && a < INT_MIN - b);
}

int main() {
  std::vector<int> vec = { /* ... */ };
  auto sum = accumulate(
    vec.begin(), vec.end(), 0, 
    [](int a, int b) 
      pre (!add_overflows(a, b))  // precondition on lambda
    {
      return a + b;
    });
  // ...
}
\end{codeblock}

While not explicitly stated in \cite{P2900R1}, we assume that assertions are already allowed on lambdas because they go into the lambda body and are allowed anywhere expressions are allowed.

\section{Discussion}
\subsection{Name lookup}

The most recent discussion of contract checks on lambdas can be found in \cite{P2388R4}:

\pagebreak

\begin{adjustwidth}{0.5cm}{0.5cm}
These features are deferred due to unresolved issues: [...] a way to express preconditions and postconditions for lambdas: name lookup is already problematic in lambdas in the face of lambda captures. This problem is pursued in \cite{P2036R1}, and until it has been solved we see no point in delaying the minimum contract support proposal.
\end{adjustwidth}

However, since that paper was published, \cite{P2036R3} has been adopted for C++23, which resolved the name lookup issues cited above. We therefore no longer see a problem with allowing precondition and postcondition checks on lambda expressions.

We propose that name lookup for entities inside precondition and postcondition checks on lambdas follow the same rules as they do for lambda trailing return types (see \cite{P2036R3}): name lookup first considers the captures of the lambda before looking further outward. Consider:

\begin{codeblock}
int i = 0;
double j = 42.0;
// ...
auto counter = [j=i] mutable pre (j >= 0) {
  return j++;
};
\end{codeblock}

In this code, the \tcode{j} in the precondition check should refer to the \tcode{j} of type \tcode{int} introduced by the init-capture, not the \tcode{j} of type \tcode{double} declared outside. This rule is most consistent with the rest of the language, and least surprising to the user.

\subsection{Captures}

Entities in the predicate of a contract check are unconditionally ODR-used (see \cite{P2900R1}), regardless of whether the contract check is evaluated or ignored. This must be so because the contract semantic is in general unknown at compile time (see \cite{P2877R0}).

ODR-use can trigger lambda captures. It follows that a contract check on a lambda can trigger a lambda capture if it ODR-uses an entity not ODR-used anywhere else. For example, the following contract check will unconditionally trigger a lambda capture:

\begin{codeblock}
auto f(int i) {
  return sizeof( [=] pre (i > 0) {} );   // captures \tcode{i} by value
}
\end{codeblock}

In this code, \tcode{f} will return \tcode{sizeof(int)} even if the semantic of the contract check is \emph{ignore}. With the contract check removed, \tcode{f} would return \tcode{1}.

We believe that this behaviour is most straightforward, most consistent with the rest of the language, and least surprising to the user: it simply falls out of the current rules for ODR-use and lambda captures. It is also consistent with \tcode{[[assume]]}, which behaves in the same way:

\begin{codeblock}
auto f(int i) {
  return sizeof( [=] { [[assume (i > 0)]]; } );   // captures \tcode{i} by value
}
\end{codeblock}

This phenomenon has been extensively discussed when \tcode{[[assume]]} was standardised for C++23 (see \cite{P1774R8} section 4.4). While there were concerns that \tcode{[[assume]]} should not be able to change the layout of a class, ultimately EWG decided it was too much of an edge case unlikely to cause problems in real code to justify complicating the rules of the language to make such a capture ill-formed. The same reasoning can be applied to contract checks.

\cite{P2932R1} proposes to make the above case --- a contract check triggering a lambda capture of an entity not otherwise captured --- ill-formed. The paper argues that allowing this violates the \emph{zero overhead principle}: adding a contract check should never lead to a different branch being taken at compile time. Otherwise, this could lead to ``heisenbugs'' (a program contains a bug, we add a contract check to find the bug, but the contract check causes the program to take another branch where the bug does not exist) and measurable performance degradations. The paper argues that the existence of cases where the addition of a contract check can cause such degradations would be a major disincentive for the adoption of Contracts in C++. Such a degradation can occur if a lambda capture inadvertently causes an expensive copy of a object:

\begin{codeblock}
std::function<int()> foo(const std::vector<S>& v)
{
  int ndx = pickIndexAtRandom(v);
  return [=]()
    pre (0 <= ndx && ndx < v.size()) // needs to capture \tcode{v}
  {
    return ndx; // Obviously we intend this to capture \tcode{ndx} by value.
  }; 
}
\end{codeblock} 

While the code above is indeed inefficient, we do not find the argument to make it ill-formed particularly convincing.  Such cases are unlikely to appear in practice, and if they do, the user gets what they asked for. It is easily avoidable (explicitly capture \tcode{v.size()}, instead of implicitly capturing the entire vector \tcode{v}; or do not capture anything at all and instead assert that \tcode{ndx} is in the proper range prior to initialising the lambda). Allowing the capture to happen is consistent with how captures work in all other parts of the lambda (the trailing return type and the body). It is also consistent with the rules for \tcode{[[assume]]}. We should let these language features combine naturally, rather than artificially complicating the language rules to micromanage each special case where the user might have got it wrong --- we do not do it elsewhere in the Standard, either.

If we were to make this case ill-formed, the user would get a highly non-obvious compiler error. In cases where the capture is intended, the user would have to add an extra line that ODR-uses the entity in question inside the lambda body (for example, casting it to \tcode{void}). This also raises the question whether lambda captures would also be ill-formed if triggered by a \tcode{contract_assert} in the lambda body rather than by a \tcode{pre} or \tcode{post}. \cite{P2932R1} does not clearly state this, but the zero overhead principle seems to imply that the answer is yes, making a \tcode{contract_assert} an odd case of an expression that cannot trigger a capture. We believe that introducing such exceptions and special cases to the basic rules of the language hurts the ergonomics and teachability of C++.

An alternative that we find entirely adequate to address this issue is to implement a compiler warning for it as a matter of QoI, as is usually done in similar cases. Consider:

\begin{codeblock}
std::map<int, Widget> map = { /* ... */ };
for (const std::pair<int, Widget>& elem : map)
  // do something with \tcode{elem}
\end{codeblock}

In this case, the user got the element type of \tcode{std::map} wrong (which is \tcode{std::pair<const int, Widget>} rather than \tcode{std::pair<int, Widget>}); this generates an unintended implicit conversion, which in turn yields a temporary object that is lifetime-extended by the \tcode{const\&}. This code compiles and works, but has a silent performance degradation due to the unnecessary conversion and object creation on every iteration of the loop. Such inefficiency is unfortunate; however, we do not add special cases to basic language rules such as range-based \tcode{for} loops, implicit conversions, or reference semantics to make these cases ill-formed. Instead, the user gets what they get, and a quality compiler or static analysis tool will issue a warning.

However, if SG21 decides that addressing the issue via warnings is not sufficient, and the zero overhead principle needs to be enforced in all cases, then a contract check triggering a lambda capture of an entity not otherwise captured needs to be ill-formed. If we were to go down that path, we propose to at least make this behaviour consistent with \tcode{[[assume]]} by making the same case ill-formed for \tcode{[[assume]]}, which can be accomplished via a Defect Report.

\section{Summary}

We propose to allow add precondition and postconditon checks on lambda expressions in the Contracts MVP \cite{P2900R1}.

We propose that name lookup for entities inside precondition and postcondition checks on lambdas follow the same rules as they do for lambda trailing return types: name lookup first considers the captures of the lambda before looking further outward.

We further propose that contract checks on lambdas should follow the usual C++ rules for ODR-use and lambda captures, like \tcode{[[assume]]} does. If, however, SG21 decides to introduce a special exception that a contract check triggering a lambda capture of an entity not otherwise captured should make the program ill-formed, in order to adhere to the zero overhead principle from \cite{P2932R1} in all possible cases, we propose that we should at least submit a Defect Report to make the same case ill-formed for an \tcode{[[assume]]} inside a lambda for consistency.

%\section*{Document history}

%\begin{itemize}
%\item \textbf{R0}, 2023-03-08: Initial version.
%\item \textbf{R1}, 20XX-XX-XX: ??
%\end{itemize}

%\section*{Acknowledgements}

%nothing yet

\renewcommand{\bibname}{References}
\bibliographystyle{abstract}
\bibliography{ref}

\end{document}
